%
% $Id: Notification.tex,v 1.2 2004-01-26 23:53:22 alphonse.bendt Exp $
%

The JacORB Notification Service is a partial implementation of
the Notification Service specified by the OMG.

\section{Runnning the Notification Service}
\label{sec:ntfy-running}

Before the JacORB Notification Service can be accessed a server
process must be started. Starting
the notification server is done by running

\cmdline{ntfy [-printIOR] [-printCorbaloc] [-writeIOR <filename>]
  [-registerName <nameID>[.<nameKind>]] [-port <oaPort>] [-channels
  <channels>] [-help]} 


\subsection{Running as a NT Service or a UNIX Daemon}
\label{sec:runn-notif-serv-1}

With a little help from the
\href{http://wrapper.tanukisoftware.org}{Java Service Wrapper} it is
easy to run the JacORB notification service as a Windows Service or as
a UNIX daemon. 

\subsubsection{Installing and Running as a NT Service}
\label{sec:windows-service}

The necessary wrapper configuration files are preconfigured in the
\texttt{JacORB/bin} directory. 

The notification service can be installed as a service by double
clicking on the \texttt{NotifyService-Install-NT.bat} batch file which
is located in the \texttt{JacORB/bin} directory.
Alternatively you can open a Command Window and then run the install
script from the command prompt. 

\begin{verbatim}
  C:\JacORB\bin>NotifyService-Install-NT.bat
  wrapper  | JacORB Notification Service installed.
\end{verbatim}

Once the service has been installed, it can be started by opening up
the Service Control Panel, selecting the service, and then pressing
the start button.

The service can also be started and stopped from within a Command
Window by using the \texttt{net start JacORB-Notify} and \texttt{net
  stop JacORB-Notify} commands, or 
by passing commands to the Wrapper.exe executable. 

The wrapper is set up to start the JacORB Notification Service
whenever the machine is rebooted.

The service can be uninstalled by running the
\texttt{NotifyService-Uninstall-NT.bat} batch file.

See the Windows specific
\href{http://wrapper.tanukisoftware.org/doc/english/launch-win.html}{wrapper
  documentation} for more details.
  

\subsubsection{Installing and Running as a UNIX Daemon}
\label{sec:inst-runn-as}

JacORB is shipped with a \texttt{sh} script which can be used to start
and stop the JacORB Notification Service controlled by the Java
Service Wrapper.

First you need to download the appropiate binary for your system from
\href{http://wrapper.tanukisoftware.org}{http://wrapper.tanukisoftware.org}.
The Java Service Wrapper is supported on Windows, Linux, Solaris, AIX,
HP-UX, Macintosh OS X, DEC OSF1, FreeBSD, and SGI Irix systems (Note:
You don't need to download anything if you are running Windows. All
necessary stuff is shipped with the JacORB distribution).

You'll need two files from the downloaded distribution. One is
the system-specific \texttt{wrapper} executable. Place it somewhere in
your searchpath. The second is the native library
\texttt{libwrapper.so} in the \texttt{\emph{wrapper-dir}/lib}
directory. This library is required by the 
wrapper executable. Place the library in the \texttt{JacORB/lib}
directory. Alternatively you can create a link to the library.

Ensure that the shell-script
\texttt{JacORB/bin/ntfy-wrapper} has the executable bit set. Note that
the sh script will attempt to create a pid file in the directory
specified by the property \texttt{PIDDIR} in the script. If
the user used to launch the Wrapper does not have permission to write
to this directory then this will result in an error. An alternative
that will work in most cases is to write the pid file to another
directory. To make this change, edit the sh  script and change the following line: 

\begin{verbatim}
PIDDIR="."
\end{verbatim}
to something more appropiate:
\begin{verbatim}
PIDDIR="/var/run"
\end{verbatim}


\paragraph{Running in the console}
\label{sec:running-console}

 The JacORB notification service  can now be run by simply executing
 \texttt{bin/ntfy-wrapper console}.

 When running using the console command, output from the notification
 service will be visible in the console. 

 The notification service can be terminated by hitting CTRL-C in the command
 window. This will cause the Wrapper to shut down the service cleanly.  

 If you omit the command the scripts prints the available commands.
 The script accepts the commands start, stop, restart and dump. The
 start, stop, and restart commands are common to most daemon scripts
 and are used to control the wrapper and the notification service  as
 a daemon process. The console 
 command will launch the wrapper in the current shell, making it
 possible to kill the application with CTRL-C. The final command,
 dump, will send a kill -3 signal to the wrapper causing the its JVM
 to do a full thread dump.  


\paragraph{Running as a Daemon Process}
\label{sec:running-as-daemon}

 The application can be run as a detatched daemon process by executing
 the script using the start  command. 

 When running using the start  command, output from the JVM will only
 be visible by viewing the \texttt{NotifyService-Wrapper.log} file
 using \texttt{tail -f NotifyService-Wrapper.log}

 Because the application is running as a detatched process, it can not
 be terminated using CTRL-C and will continue to run even if the
 console is closed. 

 To stop the application rerun the script using the stop command.

 
 \paragraph{Installing The Application To Start on Reboot}
 \label{sec:inst-appl-start}

 This is system specific. See the UNIX specific
 \href{http://wrapper.tanukisoftware.org/doc/english/launch-nix.html}{wrapper
   documentation} for instructions for some platforms.

\section{Accessing the Notification Service}
\label{sec:access-notif-serv}

Configuring a default notification service as the ORB's default is done
by adding the URL that points to the service to the properties files
\texttt{.jacorb\_properties}. A valid URL can be obtained in various ways:

\begin{enumerate}
\item By specifying the option \texttt{-printIOR} as you start the
  notification service a stringified IOR is printed out to the
  console. From there you can copy it to a useful location.

\item Usually the stringified IOR makes most sense inside a file. Use
  the option \texttt{-writeIOR <filename>} to write the IOR to the specified
  file.

\item A more compact URL can be obtained by using the
  option \texttt{-printCorbaloc}. In conjunction with the option
  \texttt{-port} you can use the simplified corbaloc: URL of the form
  \texttt{corbaloc::ip-address:port/NotificationService}. This means
  all you need to know to construct an object reference to your
  notification service is the IP address of the machine and the port
  number the server process ist listening on (the one specified using
  \texttt{-port}). 

\end{enumerate}

Add the property \texttt{ORBInitRef.NotificationService} to your
properties file. The value can be a corbaloc: URL or alternatively the
file name where you saved the IOR.

The JacORB notification service is accessed using the standard CORBA
defined interface:

\small{
\begin{verbatim}
  // get a reference to the notification service
  ORB orb = ORB.init(args, null);
  org.omg.CORBA.Object o = orb.resolve_initial_references("NotificationService");
  EventChannelFactory ecf = EventChannelHelper.narrow( o );
  IntHolder cid = new IntHolder();
  EventChannel ec = ecf.create_channel(new Property[0], new Property[0], cid);
  ...
\end{verbatim}
}

%%% Local Variables: 
%%% mode: latex
%%% TeX-master: "../ProgrammingGuide"
%%% End: 
