
This document gives an introduction to programming distributed
applications with JacORB, a free Java object request broker. JacORB
comes with full source code, a couple of CORBA Object Service
implementations, and a number of example programs.  The JacORB version
described in this document is JacORB \JacORBVersion.

This document is {\it not} an introduction to CORBA in general.
Please see \cite{Brose2001a,Siegel2000, Vinoski1997} for this purpose
and \cite{Henning1999} for more advanced issues.

\section{Project History}

JacORB originated in 1995 or 1996 at Freie Universit�t Berlin as a
small Java RPC library and a stub compiler that would process Java
interfaces. This tool was written --- most for fun and out of
curiosity --- by Boris Bokowski and Gerald Brose because at that time
no Java RMI was available. We then realized how close the Java
interface syntax was to CORBA IDL, so we wrote an IDL grammar for our
parser generator and moved to GIOP and IIOP as the transport protocol.
For a long time, JacORB was the only free (in the GNU sense)
Java/CORBA implementation avialable, and it soon enjoyed widespread
interest in both academic and industrial projects.

JacORB lived as a one-man-project for a while until a few student
projects and master theses started adding to it, most notably Reimo
Tiedemann's POA implementation, and Nicolas Noffke's Implementation
Repository and Portable Interceptor implementations. Another early
contributor was Sebastian M�ller who wrote the Appligator. A more
recent addition is Alphonse Bendt's implementation of a CORBA
Notification Services as part of his master's theses, or Andr�
Spiegel's OBV and AMI implementations. Substantial contributions to
JacORB have been added obver time by the team at PrismTech UK (Steve
Osselton, Nick Cross, Simon McQueen, Jason Courage).

JacORB continues to be used for research at Freie Universit�t Berlin,
especially in the field of security. Even though a number of people
from the core team have left FUB (Gerald, Nico, and Reimo are now with
Xtradyne Technologies, Andr� Spiegel is now a freelance developer and
consultant), the JacORB project is still seated at Freie Universit�t
Berlin, which hosts the JacORB web and CVS server. Students there can
write their master's theses on top of a solid CORBA basis, knowing
that others will benefit from their work.

\section{Support}

In addition to the best effort support that we always do on the
mailing lists, it is now possible to purchase commercial JacORB
support. Please send email to {\tt info@jacorb.com} if you want
members of the JacORB core team. Commercial support is also available
from PrismTech and OCI.


\section{Contributing --- Donations}

In essence, the early development years were entirely funded by public
research. JacORB did receive some sponsoring over the years, but not
as much as would have been desirable. A few development tasks that
would otherwise not have been possible could be payed for, but more
would have been possible --- and still is. 

If you feel that returning some of the value created by the use of
Open Source software in your company is a wise investment in the
future of that the software (maintenaince, quality improvements,
further development) in the future, then you should contact us about
donations.

Buying hardware and sending it to us is one option. It is also
possible to directly donate money to the JacORB project at Freie
Universit�t Berlin. If straight donations are difficult to get through
accounting, then you can ask us to send you an invoice for, e.g..,
CORBA consulting.

\section{Contributing --- Development}

If you want to contribute to the development of the software directly,
you should do the following:

\begin{itemize}
\item download JacORB and run the software to gain some first-hand
  expertise first 
\item read this document and other sources of CORBA documentation,
  such as \cite{Brose2001a}, and the OMG's set of specifications
  (CORBA spec., IDL/Java language mapping)
\item start reading the code
\item subscribe to the {\tt jacorb-developer} mailing list to share
  your expertise
\item contact us to get subscribed to the core team's mailing list and
  gain CVS access
\item read the coding guide line
\item contribute code and test cases
\end{itemize}

%%% Local Variables: 
%%% mode: latex
%%% TeX-master: "../ProgrammingGuide"
%%% End: 

